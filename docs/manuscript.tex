% Options for packages loaded elsewhere
\PassOptionsToPackage{unicode}{hyperref}
\PassOptionsToPackage{hyphens}{url}
%
\documentclass[
  english,
  jou]{apa7}
\usepackage{amsmath,amssymb}
\usepackage{lmodern}
\usepackage{ifxetex,ifluatex}
\ifnum 0\ifxetex 1\fi\ifluatex 1\fi=0 % if pdftex
  \usepackage[T1]{fontenc}
  \usepackage[utf8]{inputenc}
  \usepackage{textcomp} % provide euro and other symbols
\else % if luatex or xetex
  \usepackage{unicode-math}
  \defaultfontfeatures{Scale=MatchLowercase}
  \defaultfontfeatures[\rmfamily]{Ligatures=TeX,Scale=1}
\fi
% Use upquote if available, for straight quotes in verbatim environments
\IfFileExists{upquote.sty}{\usepackage{upquote}}{}
\IfFileExists{microtype.sty}{% use microtype if available
  \usepackage[]{microtype}
  \UseMicrotypeSet[protrusion]{basicmath} % disable protrusion for tt fonts
}{}
\makeatletter
\@ifundefined{KOMAClassName}{% if non-KOMA class
  \IfFileExists{parskip.sty}{%
    \usepackage{parskip}
  }{% else
    \setlength{\parindent}{0pt}
    \setlength{\parskip}{6pt plus 2pt minus 1pt}}
}{% if KOMA class
  \KOMAoptions{parskip=half}}
\makeatother
\usepackage{xcolor}
\IfFileExists{xurl.sty}{\usepackage{xurl}}{} % add URL line breaks if available
\IfFileExists{bookmark.sty}{\usepackage{bookmark}}{\usepackage{hyperref}}
\hypersetup{
  pdftitle={Increased perceptions of autonomy through choice fail to enhance motor skill retention},
  pdfauthor={Laura St.~Germain1, Allison Williams1, Noura Balbaa1, Andrew Poskus1, Olena Leshchyshen1, Keith R. Lohse2, \& Michael J. Carter1},
  pdflang={en-EN},
  pdfkeywords={Motor learning; Pre-registered; Self-controlled; Observation; Equivalence testing},
  hidelinks,
  pdfcreator={LaTeX via pandoc}}
\urlstyle{same} % disable monospaced font for URLs
\usepackage{graphicx}
\makeatletter
\def\maxwidth{\ifdim\Gin@nat@width>\linewidth\linewidth\else\Gin@nat@width\fi}
\def\maxheight{\ifdim\Gin@nat@height>\textheight\textheight\else\Gin@nat@height\fi}
\makeatother
% Scale images if necessary, so that they will not overflow the page
% margins by default, and it is still possible to overwrite the defaults
% using explicit options in \includegraphics[width, height, ...]{}
\setkeys{Gin}{width=\maxwidth,height=\maxheight,keepaspectratio}
% Set default figure placement to htbp
\makeatletter
\def\fps@figure{htbp}
\makeatother
\setlength{\emergencystretch}{3em} % prevent overfull lines
\providecommand{\tightlist}{%
  \setlength{\itemsep}{0pt}\setlength{\parskip}{0pt}}
\setcounter{secnumdepth}{-\maxdimen} % remove section numbering
% Make \paragraph and \subparagraph free-standing
\ifx\paragraph\undefined\else
  \let\oldparagraph\paragraph
  \renewcommand{\paragraph}[1]{\oldparagraph{#1}\mbox{}}
\fi
\ifx\subparagraph\undefined\else
  \let\oldsubparagraph\subparagraph
  \renewcommand{\subparagraph}[1]{\oldsubparagraph{#1}\mbox{}}
\fi
% Manuscript styling
\usepackage{upgreek}
\captionsetup{font=singlespacing,justification=justified}

% Table formatting
\usepackage{longtable}
\usepackage{lscape}
% \usepackage[counterclockwise]{rotating}   % Landscape page setup for large tables
\usepackage{multirow}		% Table styling
\usepackage{tabularx}		% Control Column width
\usepackage[flushleft]{threeparttable}	% Allows for three part tables with a specified notes section
\usepackage{threeparttablex}            % Lets threeparttable work with longtable

% Create new environments so endfloat can handle them
% \newenvironment{ltable}
%   {\begin{landscape}\begin{center}\begin{threeparttable}}
%   {\end{threeparttable}\end{center}\end{landscape}}
\newenvironment{lltable}{\begin{landscape}\begin{center}\begin{ThreePartTable}}{\end{ThreePartTable}\end{center}\end{landscape}}

% Enables adjusting longtable caption width to table width
% Solution found at http://golatex.de/longtable-mit-caption-so-breit-wie-die-tabelle-t15767.html
\makeatletter
\newcommand\LastLTentrywidth{1em}
\newlength\longtablewidth
\setlength{\longtablewidth}{1in}
\newcommand{\getlongtablewidth}{\begingroup \ifcsname LT@\roman{LT@tables}\endcsname \global\longtablewidth=0pt \renewcommand{\LT@entry}[2]{\global\advance\longtablewidth by ##2\relax\gdef\LastLTentrywidth{##2}}\@nameuse{LT@\roman{LT@tables}} \fi \endgroup}

% \setlength{\parindent}{0.5in}
% \setlength{\parskip}{0pt plus 0pt minus 0pt}

% \usepackage{etoolbox}
\makeatletter
\patchcmd{\HyOrg@maketitle}
  {\section{\normalfont\normalsize\abstractname}}
  {\section*{\normalfont\normalsize\abstractname}}
  {}{\typeout{Failed to patch abstract.}}
\patchcmd{\HyOrg@maketitle}
  {\section{\protect\normalfont{\@title}}}
  {\section*{\protect\normalfont{\@title}}}
  {}{\typeout{Failed to patch title.}}
\makeatother
\shorttitle{Autonomy-support not beneficial for learning}
\keywords{Motor learning; Pre-registered; Self-controlled; Observation; Equivalence testing\newline\indent Word count: X}
\usepackage{dblfloatfix}


\usepackage{csquotes}
\raggedbottom
\leftheader{St. Germain et al. 2021}
\usepackage{lineno}
\pagewiselinenumbers
\ifxetex
  % Load polyglossia as late as possible: uses bidi with RTL langages (e.g. Hebrew, Arabic)
  \usepackage{polyglossia}
  \setmainlanguage[]{english}
\else
  \usepackage[main=english]{babel}
% get rid of language-specific shorthands (see #6817):
\let\LanguageShortHands\languageshorthands
\def\languageshorthands#1{}
\fi
\ifluatex
  \usepackage{selnolig}  % disable illegal ligatures
\fi
\newlength{\cslhangindent}
\setlength{\cslhangindent}{1.5em}
\newlength{\csllabelwidth}
\setlength{\csllabelwidth}{3em}
\newenvironment{CSLReferences}[2] % #1 hanging-ident, #2 entry spacing
 {% don't indent paragraphs
  \setlength{\parindent}{0pt}
  % turn on hanging indent if param 1 is 1
  \ifodd #1 \everypar{\setlength{\hangindent}{\cslhangindent}}\ignorespaces\fi
  % set entry spacing
  \ifnum #2 > 0
  \setlength{\parskip}{#2\baselineskip}
  \fi
 }%
 {}
\usepackage{calc}
\newcommand{\CSLBlock}[1]{#1\hfill\break}
\newcommand{\CSLLeftMargin}[1]{\parbox[t]{\csllabelwidth}{#1}}
\newcommand{\CSLRightInline}[1]{\parbox[t]{\linewidth - \csllabelwidth}{#1}\break}
\newcommand{\CSLIndent}[1]{\hspace{\cslhangindent}#1}

\title{Increased perceptions of autonomy through choice fail to enhance motor skill retention}
\author{Laura St.~Germain\textsuperscript{1}, Allison Williams\textsuperscript{1}, Noura Balbaa\textsuperscript{1}, Andrew Poskus\textsuperscript{1}, Olena Leshchyshen\textsuperscript{1}, Keith R. Lohse\textsuperscript{2}, \& Michael J. Carter\textsuperscript{1}}
\date{}


\authornote{

\addORCIDlink{Michael J. Carter}{0000-0002-0675-4271}

We have outlined author contributions using CRediT (Contributor Roles Taxonomy - \url{https://casrai.org/credit/}).

The authors made the following contributions. Laura St.~Germain: Conceptualization, Data curation, Formal analysis, Investigation - Performed the experiment, Methodology, Project administration, Software - Task Programming, Validation, Visualization, Writing - Original Draft Preparation, Writing - Review \& Editing; Allison Williams: Investigation - Performed the experiment, Writing - Original Draft Preparation, Writing - Review \& Editing; Noura Balbaa: Investigation - Performed the experiment, Writing - Original Draft Preparation, Writing - Review \& Editing; Andrew Poskus: Investigation - Performed the experiment, Writing - Review \& Editing; Olena Leshchyshen: Investigation - Performed the experiment, Writing - Review \& Editing; Keith R. Lohse: Conceptualization, Data curation, Formal analysis, Methodology, Validation, Writing - Original Draft Preparation, Writing - Review \& Editing; Michael J. Carter: Conceptualization, Data curation, Formal analysis, Funding acquisition, Methodology, Project administration, Resources, Software - Task Programming, Supervision, Validation, Visualization, Writing - Original Draft Preparation, Writing - Review \& Editing.

Correspondence concerning this article should be addressed to Michael J. Carter, 1280 Main Street West, Ivor Wynne Centre Room 203, McMaster University, Hamilton ON Canada, L8S 4K1. E-mail: \href{mailto:michaelcarter@mcmaster.ca}{\nolinkurl{michaelcarter@mcmaster.ca}}

}

\affiliation{\vspace{0.5cm}\textsuperscript{1} Department of Kinesiology, McMaster University\\\textsuperscript{2} Program in Physical Therapy, Washington University School of Medicine in Saint Louis}

\abstract{
There has been growing research interest in the effects that motivation plays in motor learning, and specifically how different manipulations that affect autonomy, perceptions of competence, and social relatedness may directly benefit the learning process. In the present study, we present a well-powered (80\%, N = 150) pre-registered manipulation of autonomy support by providing learners with choice during the practice of a speed cup-stacking skill. One group was given control over when a video demonstration was provided and the speed with which the demonstration was played. A yoked control group received an identical schedule of the demonstrations, but no choice (as their schedule was matched to a participant with choice). Critically, we also address a gap in the literature by adding a yoked control group who was explicitly told that they were being denied choice and that their schedule was chosen by another participant. (In the traditional yoked group, participants are merely told the schedule is determined in advance.) We found no statistically significant differences between groups in their learning of the cup-stacking skill, despite finding evidence that providing choice increased perceived autonomy (internally validating the manipulation). The two-one-side-test procedure further showed that although the groups were not statistically equivalent, the effect size is likely too small to practically study the effects of autonomy-support through choice in most motor learning labs. The current study not only adds to a growing body of research that questions the direct causal role that autonomy-support has on learning, but also the robustness of the so-called self-controlled learning advantage.
}



\begin{document}
\maketitle

A popular recommendation in recent years for creating an effective environment for motor skill learning has been to allow the learner to take control over an element of their practice that is traditionally controlled by a coach, therapist, or teacher (Sanli et al., 2013; Ste-Marie et al., 2019). This recommendation is based on the consistent finding that participants in a self-controlled (i.e., choice) group perform with higher proficiency compared to participants in a yoked (i.e., control) group on delayed retention and/or transfer tests. Participants in the yoked group do not experience the same choice opportunity provided to those in the self-controlled group. Instead, they are linked to a self-controlled participant and experience this participant's self-selected practice schedule. This so-called self-controlled learning advantage has been shown when participants are given the opportunity to schedule task difficulty \emph{()}, the order that multiple tasks are practiced \emph{()}, the frequency of watching a modeled demonstration \emph{()}, and when to receive augmented feedback \emph{()}.

Over the years, this manipulation has been described using a variety of names \emph{()}, but more recently it has been subsumed by autonomy-support. In fact, autonomy-support is considered such a robust learning variable that it is one of three key pillars in the recently proposed ``OPTIMAL'' theory of motor learning \emph{()}. Wulf and Lewthwaite argued that providing learners with opportunities for choice---considered an autonomy-supportive practice manipulation---can facilitate motor performance and learning by enhancing learner's expectancies for success, and by allowing the learner to maintain their attentional focus on the task by reducing the need for self-regulatory activity \emph{()}. In other words, these psychological and attentional benefits, and concomitant increases in performance and learning are a by-product of experiencing choice opportunities during practice. Overall, autonomy-support is seen as a means to efficient goal-action coupling \emph{()} and also links the ``OPTIMAL'' theory with Self-Determination theory \emph{()}.

Despite its prominent role within the ``OPTIMAL'' theory of motor learning, there are numerous reasons to doubt the importance of autonomy-support through the provision of choice during practice. First, if the benefits are the result of having opportunities for choice then learning differences should not emerge between different self-controlled groups. However, in experiments where different groups of participants have choice over their feedback schedule, such learning differences have been found when this choice is made after rather than before a performance attempt \emph{()}, when different criteria for success are provided to participants \emph{()}, and when the absolute number of feedback choice opportunities are limited compared to unlimited at the outset of practice \emph{()}. Second, there has been little-to-no support for the notion that practicing in a self-controlled group is perceived as more autonomy-supportive than being in a yoked group. Ste-Marie and colleagues \emph{()}, for example, had participants in the self-controlled and yoked groups complete the perceived choice subscale of the Intrinsic Motivation Inventory and failed to find the expected effect of higher self-reported scores during practice in the self-controlled group. Similar findings have been reported by others \emph{()}; however, McKay \& Ste-Marie \emph{()} recently found that practicing in a self-controlled group was perceived as more autonomy-supportive than practicing the same task in a yoked group. This increased perceived autonomy-support did not, however, translate into enhanced learning compared to the yoked group. Although the available literature does not provide substantial evidence that self-controlled practice is autonomy-supportive, this does not necessarily mean such a manipulation is not autonomy-supportive \emph{()}. For instance, this effect may in fact be quite small and require much larger sample sizes to detect \emph{()} than those commonly used in motor learning experiments \emph{()}.

There are at least two other methodological issues that warrant consideration. The first of these is that in self-controlled motor learning experiments participants in the self-controlled group are usually given choice over a single component (e.g., feedback or when to watch a modeled demonstration) of their practice and participants in the yoked group are not given choice over this component. However, within the context of practice itself there are many other opportunities for choice that participants may explore, independent of their assigned group. Target-based tasks such as basketball free throws or bean-bag tossing are a popular choice in motor learning experiments and have been used in self-controlled learning experiments \emph{()}. While a participant in the yoked group may not be permitted choice over their feedback schedule (or some other practice variable) with such tasks, this does not preclude them from being able to experience choice opportunities---and thus experience autonomy-support---when trying different throwing techniques, speeds, or release points. Thus, labeling yoked groups as being devoid of choice opportunities may be a misnomer. The other consideration relates to the instructions that are provided to participants in the yoked group. In the context of feedback\footnote{While feedback is used in this example, this issue surrounding instructions is also relevant to other practice variables commonly used in self-controlled learning experiments.}, these participants are typically informed that during practice they may or may not receive feedback after a given trial \emph{()}. This means that participants in yoked groups are not even aware that they have been denied an opportunity for choice, nor that their feedback schedule was created by another participant who was given choice over when feedback was or was not provided. Either of these in isolation, or both simultaneously could contribute to the consistent finding that participants in self-controlled and yoked groups report similar perceived autonomy scores when asked about their opportunities for choices with respect to the motor task \emph{()} or about their practice environment in general \emph{()}.

Here we investigated the effects of making participants in a yoked group explicitly aware of not only being denied opportunities for choice over their frequency of watching and the playback speed of video demonstrations, but also that the schedule they would experience during practice was created by another participant in the experiment. The addition of this novel yoked group allowed us to address one of the previously identified methodological limitations regarding experimental group instructions of previous self-controlled research. We compared the performance of this explicit yoked group with traditional self-controlled and yoked groups on a speed cup-stacking task in practice and in a delayed retention test.

\hypertarget{methods}{%
\section{Methods}\label{methods}}

We report how we determined our sample size, all data exclusions (if any), all manipulations, and all measures in the study \emph{()}. The experimental design and analyses were preregistered using AsPredicted.org and is available here: \textbf{ADD URL}.

\hypertarget{participants}{%
\subsection{Participants}\label{participants}}

One-hundred and fifty participants completed the experiment. Sample size was determined by an a priori power calculation based on our smallest comparison of interest. An early estimate for the effect of self-controlled over yoked practice was a Hedges' \(g = 0.63\) \emph{()} and while planning this experiment this effect was estimated to be \(g = 0.52\) \emph{()}. Based on this effect, a positive correlation of \(r = 0.6\) between retention and pre-test as the covariate, and 80\% power to detect a difference between the self-controlled and traditional yoked groups, the required number of participants was 31 per group. Considering the novelty of our explicit yoked group, we assumed a smaller effect, \(g = 0.4\), between it and the traditional yoked group. With the same parameters as above, this resulted in our final sample of 50 participants per group. Participants completed the experiment in either the Self-Controlled group \((M_{age} = 18.0, SD = 0.34)\), the Traditional Yoked group \((M_{age} = 19.5, SD = 1.89)\), or the Explicit Yoked group \((M_{age} = 19.2, SD = 1.55)\). We collected the Self-Controlled group first as their self-selected observation schedule was required for the yoking procedure for the two other groups. Once the Self-Controlled group had been collected, participants were randomly assigned to one of the two Yoked groups. All participants provided written informed consent approved by and conducted in accordance with the University's Research Ethics Board. Participants received either \$15 or a course bonus for their participation.

\hypertarget{material}{%
\subsection{Material}\label{material}}

Participants were tasked with learning the 3-6-3 speed cup stacking sequence based on the rules of the World Sport Stacking Association \href{https://www.thewssa.com/}{(https://www.thewssa.com/)}. The sequence consisted of an upstack phase and a downstack phase using official Speed Stacks cups \href{https://www.speedstacks.com/}{(https://www.speedstacks.com/)}. Participants performed the task using both their hands and had to complete an upstacking and a downstacking phase. The upstacking phase began by completing the first 3 cup pyramid, followed by the 6 cup pyramid, and then the other 3 cup pyramid. The downstacking phase began by returning to and collapsing the 3 cup pyramid that was upstacked first, then the 6 cup pyramid, and finally the last 3 cup pyramid.

\hypertarget{procedure}{%
\subsection{Procedure}\label{procedure}}

Participants completed two data collection sessions separated by approximately 24 hours. Session 1 consisted of a pre-test and an acquisition phase. Session 2 consisted of a delayed retention test. At the start of each phase of the experiment, all participants received phase-specific instructions \textbf{(see Table 1)}. Group specific instructions were provided prior to the acquisition phase. The instructions appeared on a 22-inch computer monitor (1920x1080 resolution) positioned to the right of the participant. Participants followed along as the instructions were read aloud by the researcher.

Each trial began with participants standing at a standard height table with their hands on marked positions on the table in front of them. The 12 cups were located in upside down stacks of 3-6-3 in front of the participant. Following a ``Get Ready!'' prompt displayed for 1 s and a constant foreperiod of 1 s, an audiovisual ``Go-signal'' (green square and a beep tone) was presented. Participants were instructed to start stacking as quickly as possible following the ``Go-signal'' as its presentation initiated the timer. Once the upstack and downstack phases were completed, participants were instructed to press the spacebar on a keyboard located in front of them to stop the timer. If an error occurred (e.g., only completed the upstack phase then stopped the timer, forgot to hit the timer, etc), the experimenter recorded the trial number for later removal. The pre-test and delayed retention test both consisted of five no-feedback trials. The acquisition phase had 25 trials and was the only phase where the video demonstration could be watched based on group assignment. Participants in the Self-Controlled group could decide at the start of each trial if they wanted to watch the video demonstration. If they chose to watch the video, they were then asked whether they wanted to watch it in real-time or slow motion (35\% of real-time). Participants in the Traditional Yoked and Explicit Yoked groups received the demonstration schedule created by a participant in the Self-Controlled group with the exception that participants in the Explicit Yoked group were made aware that this schedule was created by another participant \textbf{(see Table 1)}. Feedback about the participant's stacking time was provided after every trial in acquisition. The timeline of a typical trial is illustrated in \textbf{Figure 1}.

To test predictions based on the ``OPTIMAL'' theory regarding the role of motivation, enhanced expectancies, and autonomy-support, participants completed the interest/enjoyment and perceived competence subscales from the Intrinsic Motivation Inventory \emph{()} and a custom scale regarding choice used in previous self-controlled motor learning experiments \emph{()}. The order of questions from each scale were randomized and each question was rated using a 7-point Likert scale. Participants answered these questions after the pre-test, after trials five and 25 of acquisition, and before the delayed retention test.

A custom LabVIEW (National Instruments Inc.) program was created that controlled the presentation of all instructions, the video demonstrations, the timing of the experimental protocol, and recorded and stored the data for later analysis.

\hypertarget{data-analysis}{%
\subsection{Data analysis}\label{data-analysis}}

Our primary outcome measure was stacking time (i.e., response time) in seconds. Trials recorded as errors \((76/5250 = 1.45\%)\) during data collection were manually removed prior to data analysis. For each participant, pre-test and delayed retention trials were aggregated into one block of five trials and acquisition was aggregated into five blocks of five trials. Significance level was set to 0.05 for all statistical tests. Effect sizes for omnibus tests are reported using generalized eta squared \((\eta^2_{G})\) or eta squared \((\eta^2)\). Hedges' \(g\) values were calculated to gauge the effect size of the post hoc comparisons using Holm's correction. A Cook's distance of \(\geq 1\) was used to identify any influential cases.

\hypertarget{results}{%
\section{Results}\label{results}}

\hypertarget{pre-registered-analysis}{%
\subsection{Pre-registered analysis}\label{pre-registered-analysis}}

To test whether delayed retention was differentially impacted by the experimental group (i.e., Self-Controlled, Traditional Yoked, Explicit Yoked) experienced during acquisition, we performed a one-way ANCOVA controlling for pre-test \textbf{(Fig 2C)}. As can be seen, the Self-Controlled group \((M = 9.99, \,95\% \,CI = [9.68, \,10.31])\), the Traditional Yoked group \((M = 10.18, \,95\% \,CI = [9.86, \,10.50])\), and the Explicit Yoked group \((M = 10.12, \,95\% \,CI = [9.81, \,10.44])\) all had similar stacking times in retention (means are shown as the adjusted means controlling for pre-test). The effect of Group was not significant, \(F(2, 146) = .335, \,p = .716, \,\eta^2 = .002\).

\#\#Non pre-registered analyses
\#\#\#Traditional self-controlled learning advantage
To investigate whether a traditional self-controlled learning advantage existed---that is a comparison between the Self-Controlled and Traditional Yoked groups---we analyzed the non-adjusted retention scores using a Welch's t-test. The analysis revealed that the Self-Controlled group \((M = 9.77, \,95\% \,CI = [9.55, \,9.99])\) and the Traditional Yoked group \((M = 10.20, \,95\% \,CI = [9.92, \,10.49])\) were not statistically different, \(t(88.10) = 1.45, \,p = .15, \,g = .29\).

\hypertarget{discussion}{%
\section{Discussion}\label{discussion}}

\newpage

\hypertarget{references}{%
\section{References}\label{references}}

\begingroup
\setlength{\parindent}{-0.5in}
\setlength{\leftskip}{0.5in}

\hypertarget{refs}{}
\begin{CSLReferences}{1}{0}
\leavevmode\hypertarget{ref-sanli2013}{}%
Sanli, E. A., Patterson, J. T., Bray, S. R., \& Lee, T. D. (2013). Understanding self-controlled motor learning protocols through the self-determination theory. \emph{Frontiers in Psychology}, \emph{3}. \url{https://doi.org/10.3389/fpsyg.2012.00611}

\leavevmode\hypertarget{ref-ste-marie2019}{}%
Ste-Marie, D. M., Carter, M. J., \& Yantha, Z. D. (2019). \emph{Self-controlled learning: Current findings, theoretical perspectives, and future directions} (3rd ed.). Routledge.

\end{CSLReferences}

\endgroup


\end{document}
