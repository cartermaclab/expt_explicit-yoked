% Taken from https://github.com/mschroen/review_response_letter
% GNU General Public License v3.0

\documentclass[final]{article}

\usepackage[includeheadfoot,top=20mm, bottom=20mm, footskip=2.5cm]{geometry}

% Typography
\usepackage[T1]{fontenc}
\usepackage{times}
%\usepackage{mathptmx} % math also in times font
\usepackage{amssymb,amsmath}
\usepackage{microtype}
\usepackage[utf8]{inputenc}

% Misc
\usepackage{graphicx}
\usepackage[hidelinks]{hyperref} %textopdfstring from pandoc
\usepackage{soul} % Highlight using \hl{}

% Table

\usepackage{adjustbox} % center large tables across textwidth by surrounding tabular with \begin{adjustbox}{center}
\renewcommand{\arraystretch}{1.5} % enlarge spacing between rows
\usepackage{caption}
\captionsetup[table]{skip=10pt} % enlarge spacing between caption and table

% Section styles

\usepackage{titlesec}
\titleformat{\section}{\normalfont\large}{\makebox[0pt][r]{\bf \thesection.\hspace{4mm}}}{0em}{\bfseries}
\titleformat{\subsection}{\normalfont}{\makebox[0pt][r]{\bf \thesubsection.\hspace{4mm}}}{0em}{\bfseries}
\titlespacing{\subsection}{0em}{1em}{-0.3em} % left before after

% Paragraph styles

\setlength{\parskip}{0.6\baselineskip}%
\setlength{\parindent}{0pt}%

% Quotation styles

\usepackage{framed}
\let\oldquote=\quote
\let\endoldquote=\endquote
\renewenvironment{quote}{\begin{fquote}\advance\leftmargini -2.4em\begin{oldquote}}{\end{oldquote}\end{fquote}}

% \usepackage{xcolor}
\newenvironment{fquote}
  {\def\FrameCommand{
	\fboxsep=0.6em % box to text padding
	\fcolorbox{black}{white}}%
	% the "2" can be changed to make the box smaller
    \MakeFramed {\advance\hsize-2\width \FrameRestore}
    \begin{minipage}{\linewidth}
  }
  {\end{minipage}\endMakeFramed}

% Table styles

\let\oldtabular=\tabular
\let\endoldtabular=\endtabular
\renewenvironment{tabular}[1]{\begin{adjustbox}{center}\begin{oldtabular}{#1}}{\end{oldtabular}\end{adjustbox}}


% Shortcuts

%% Let textbf be both, bold and italic
%\DeclareTextFontCommand{\textbf}{\bfseries\em}

%% Add RC and AR to the left of a paragraph
%\def\RC{\makebox[0pt][r]{\bf RC:\hspace{4mm}}}
%\def\AR{\makebox[0pt][r]{AR:\hspace{4mm}}}

%% Define that \RC and \AR should start and format the whole paragraph
\usepackage{suffix}
\long\def\RC#1\par{\makebox[0pt][r]{\bf RC:\hspace{4mm}}{\bf #1}\par\makebox[0pt][r]{AR:\hspace{10pt}}} %\RC
\WithSuffix\long\def\RC*#1\par{{\bf #1}\par} %\RC*
% \long\def\AR#1\par{\makebox[0pt][r]{AR:\hspace{10pt}}#1\par} %\AR
\WithSuffix\long\def\AR*#1\par{#1\par} %\AR*


%%%
%DIF PREAMBLE EXTENSION ADDED BY LATEXDIFF
%DIF UNDERLINE PREAMBLE %DIF PREAMBLE
\RequirePackage[normalem]{ulem} %DIF PREAMBLE
\RequirePackage{color} %DIF PREAMBLE
\definecolor{offred}{rgb}{0.867, 0.153, 0.153} %DIF PREAMBLE
\definecolor{offblue}{rgb}{0.0705882352941176, 0.168627450980392, 0.717647058823529} %DIF PREAMBLE
\providecommand{\DIFdel}[1]{{\protect\color{offred}\sout{#1}}} %DIF PREAMBLE
\providecommand{\DIFadd}[1]{{\protect\color{offblue}\uwave{#1}}} %DIF PREAMBLE
%DIF SAFE PREAMBLE %DIF PREAMBLE
\providecommand{\DIFaddbegin}{} %DIF PREAMBLE
\providecommand{\DIFaddend}{} %DIF PREAMBLE
\providecommand{\DIFdelbegin}{} %DIF PREAMBLE
\providecommand{\DIFdelend}{} %DIF PREAMBLE
%DIF FLOATSAFE PREAMBLE %DIF PREAMBLE
\providecommand{\DIFaddFL}[1]{\DIFadd{#1}} %DIF PREAMBLE
\providecommand{\DIFdelFL}[1]{\DIFdel{#1}} %DIF PREAMBLE
\providecommand{\DIFaddbeginFL}{} %DIF PREAMBLE
\providecommand{\DIFaddendFL}{} %DIF PREAMBLE
\providecommand{\DIFdelbeginFL}{} %DIF PREAMBLE
\providecommand{\DIFdelendFL}{} %DIF PREAMBLE
%DIF END PREAMBLE EXTENSION ADDED BY LATEXDIFF

% Fix pandoc related tight-list error
\providecommand{\tightlist}{%
  \setlength{\itemsep}{0pt}\setlength{\parskip}{0pt}}

% Add task difficulty and assignment commands from https://github.com/cdc08x/letter-2-reviewers-LaTeX-template
\usepackage[usenames,dvipsnames]{xcolor}
\usepackage{ifdraft}

\newcommand{\TaskEstimationBox}[2]{%
\ifoptiondraft{\parbox{1.0\linewidth}{\hfill \hfill {\colorbox{#2}{\color{White} \textbf{#1}}}}}%
{}%
}
%
\def\WorkInProgress {\TaskEstimationBox{Work in progress}{Cyan}}
\def\AlmostDone {\TaskEstimationBox{Almost there}{NavyBlue}}
\def\Done {\TaskEstimationBox{Done}{Blue}}
%
\def\NotEstimated {\TaskEstimationBox{Effort not estimated}{Gray}}
\def\Easy {\TaskEstimationBox{Feasible}{ForestGreen}}
\def\Medium {\TaskEstimationBox{Medium effort}{Orange}}
\def\TimeConsuming {\TaskEstimationBox{Time-consuming}{Bittersweet}}
\def\Hard {\TaskEstimationBox{Infeasible}{Black}}
%
\newcommand{\Assignment}[1]{
%
\ifoptiondraft{%
\vspace{.25\baselineskip} \parbox{1.0\linewidth}{\hfill \hfill \vspace{.25\baselineskip} \normalfont{Assignment:} \normalfont{\textbf{#1}}}%
}{}%
}




\newlength{\cslhangindent}
\setlength{\cslhangindent}{1.5em}
\newlength{\csllabelwidth}
\setlength{\csllabelwidth}{3em}
\newenvironment{CSLReferences}[2] % #1 hanging-ident, #2 entry spacing
 {% don't indent paragraphs
  \setlength{\parindent}{0pt}
  % turn on hanging indent if param 1 is 1
  \ifodd #1 \everypar{\setlength{\hangindent}{\cslhangindent}}\ignorespaces\fi
  % set entry spacing
  \ifnum #2 > 0
  \setlength{\parskip}{#2\baselineskip}
  \fi
 }%
 {}
\usepackage{calc}
\newcommand{\CSLBlock}[1]{#1\hfill\break}
\newcommand{\CSLLeftMargin}[1]{\parbox[t]{\csllabelwidth}{#1}}
\newcommand{\CSLRightInline}[1]{\parbox[t]{\linewidth - \csllabelwidth}{#1}\break}
\newcommand{\CSLIndent}[1]{\hspace{\cslhangindent}#1}

\begin{document}

{\Large\bf Author response to reviews of}\\[1em]
Manuscript XHP-2021-1912\\ \\
{\Large Increased perceptions of autonomy through choice fail to enhance motor skill retention}\\[1em]
{Laura St.~Germain, Allison Williams, Noura Balbaa, Andrew Poskus, Olena Leshchyshen, Keith R. Lohse \& Michael J. Carter}\\
{submitted to \it Journal of Experimental Psychology: Human Perception and Performance }\\
\hrule

\hfill {\bfseries RC:} \textbf{\textit{Reviewer Comment}}\(\quad\) AR: Author Response \(\quad\square\) Manuscript text

\vspace{2em}

Jacqueline Clare Snow, PhD\\
Associate Editor\\
\emph{Journal of Experimental Psychology: Human Perception and Performance}

Dear Dr.~Snow,

Thank you for taking the time to consider our manuscript for publication at \emph{Journal of Experimental Psychology: Human Perception and Performance}. We have read the comments provided by the Reviewers and have revised the manuscript accordingly. We thank the Reviewers for their helpful and insightful comments. Below we provide a point-by-point reply to the Reviewer comments (in bold-italic) and our responses in normal text. Any substantial changes that have been made to the original submission are also included in the response letter (in a text box) and are in red font in the revised manuscript.

Kind regards,

Laura St.~Germain\\
Allison Williams\\
Noura Balbaa\\
Andrew Poskus\\
Olena Leshchyshen\\
Keith R. Lohse\\
Michael J. Carter

\hypertarget{reviewer-1}{%
\section{Reviewer \#1}\label{reviewer-1}}

\RC{\emph{The authors provide a well-thought-out and clearly presented research design with clear theoretical implications. In particular, the analyses were very appropriate and should serve as an example for others conducting similar work. Though I carefully reviewed the manuscript, I do not have any specific suggested changes or questions for the authors. Though certainly not necessary for publication, the authors may wish consider the work of Katz and Assor (2007) surrounding choice and motivation from a Self Determination Theory perspective, as this seems to be relevant to the present findings. 
}}

Thank you for the positive comments about our manuscript and for sharing the reference to Katz and Assor (2007).

\hypertarget{reviewer-2}{%
\section{Reviewer \#2}\label{reviewer-2}}

\RC{\emph{The authors submitted a well written manuscript reporting a pre-registered experiment that investigated the effect of awareness of being denied control over two relevant variables of the practice environment (i.e., frequency and speed of video demonstration) as participants learned the speed cup-stacking task. Results revealed that participants allowed to choose when to watch and the speed of the video demonstrations (i.e., self-controlled group) showed higher perceived autonomy scores when compared to yoked participants who received the demonstrations in matched schedule to a self-controlled counterpart and were either explicitly aware of this information (Explicit Yoked) or not (Traditional Yoked). However, the authors failed to replicate the so-called self-control learning benefit as no differences between groups were found in the retention test. Additionally, groups did not differ in self-reported measures of perceived competence and intrinsic motivation.}

\emph{Overall, I think this paper can be a valuable addition to the literature. The study is methodologically strong, it has one of the largest sample sizes reported in the motor learning literature, the authors pre-registered the main analyses, carried out a power calculation, and equivalence test, and heartfully embraced open science practices (I was able to easily reproduce the analyses/figures reported in the paper). However, I do have a few concerns/recommendations that, if addressed, can help to further improve the paper. 
}}

Thank your for the positive general comments about our manuscript. We are pleased to hear that you were able to reproduce the analyses and figures. Below are the responses to your specific comments. We have split the first comment into smaller chunks to facilitate our response.

\RC{\emph{My major concern is how the study's rationale and relevance are framed. The authors start by questioning the role played by autonomy support in practice conditions wherein provision of choice is manipulated. They argued that "if the benefits are the result of having opportunities for choice then learning differences should not emerge between different self-controlled groups" [Page 5, line 7]. However, this claim disregards another possible explanation for the self-control learning benefit, namely information processing factors (Barros et al., 2019).
}}

We agree with the reviewer that an information-processing explanation has also been forwarded to account for the self-controlled learning advantages. We did not not specifically refer to it in the introduction as this experiment was not designed to dissociate between the motivational and the information-processing explanations. We have removed the quoted text in the revised submission and have added the following as a footnote to acknowledge the information-processing view \textbf{(see Page XX)}:

\begin{quote}
It should be noted that other researchers have instead presented an information-processing explanation for the self-controlled learning advantages (see Ste-Marie et al., 2019 for a recent discussion of the motivational and information-processing explanations). We acknowledge this view here; however, unlike previous experiments (e.g., Barros, Yantha, Carter, Hussien, \& Ste-Marie, 2019; Carter, Carlsen, \& Ste-Marie, 2014; Carter \& Ste-Marie, 2017b, 2017a; Couvillion, Bass, \& Fairbrother, 2020; Woodard \& Fairbrother, 2020) the current experiment was not designed to test between explanations. We focused on the motivational explanation as this view has garnered more attention due the ``OPTIMAL'' theory of motor learning.
\end{quote}

\RC{\emph{Second, they state that "there has been little-to-no support for the notion that practicing in a self-controlled group is perceived as more autonomy-supportive than being in a yoked group" [Page 5, line 15]. Although this statement has been supported by previous studies such as the one cited in the paper (Ste-Marie et al., 2013), group differences regarding levels of perceived autonomy have been found in recent studies (McKay \& Ste-Marie, 2020- also cited in the paper). The evidence, therefore, as presented in the paper seems mixed. Moreover, another explanation for the lack of group differences might have to do with the fact that increased levels of perceived autonomy have been assumed instead of directly measured in self-control studies. Thus, it becomes unclear whether the little-to-no support for the notion that practicing in a self-controlled group is perceived as more autonomy-supportive comes from a lack of differences when there is a measure of perceived autonomy or, rather, from a lack of studies assessing how provision of choice affects perception of autonomy. The authors then proceed to claim that "Although the available literature does not provide substantial evidence that self-controlled practice is autonomy-supportive, this does not necessarily mean such [extra 'a' here] manipulation is not autonomy-supportive. For instance, this effect may in fact be quite small and require much larger sample sizes to detect than those commonly used in motor learning experiments" [Page 5, line 25], which is a fair claim to make considering the methodological concerns surrounding motor learning studies (Lohse et al., 2016). However, the way this argument was structured led to the notion that it is still unclear whether self-control conditions lead to higher levels of perceived autonomy, circling back to my previous argument that the evidence is somewhat mixed and might be due to the lack of studies directly measuring perceived autonomy.
}}

We agree with many of the points raised here. We attempted to capture the mixed nature of findings by presenting experiments that failed to find an effect and the one that did find the expected effect. We restricted our discussion of past work not including measures in the Discussion as the true impact of this unknown and speculative. Based on your comments, we have restructured part of the Introduction to improve the clarity of our arguments \textbf{(see Pages XX, Lines XX)}:

\begin{quote}
Despite its prominent role as a robust and generalizable learning variable in the ``OPTIMAL'' theory (p.~1393), there is considerable ambiguity surrounding whether the provision of choice is in fact an autonomy-supportive manipulation. First, the notion that practicing in a self-controlled group is actually more autonomy-supportive than a yoked group has primarily been assumed (e.g., Abdollahipour, Palomo Nieto, Psotta, \& Wulf, 2017; Chua, Wulf, \& Lewthwaite, 2018; Lewthwaite, Chiviacowsky, Drews, \& Wulf, 2015; Wulf, Lewthwaite, Cardozo, \& Chiviacowsky, 2018) rather than supported empirically. Second, when researchers have included measures related to perceptions of autonomy the data is mixed. For example, Ste-Marie, Vertes, Law, and Rymal (2013) did not find the expected effect of higher perceptions of autonomy during practice in a self-controlled group as compared to a yoked group. Similar outcomes have been reported by others (e.g., Barros et al., 2019; Carter \& Ste-Marie, 2017b; McKay \& Ste-Marie, 2020a). In contrast, McKay and Ste-Marie (2020b) recently found that practicing in a self-controlled group was perceived as more autonomy-supportive than practicing the same task in a yoked group. However, the higher perceived autonomy scores did not translate into enhanced motor performance or learning in the self-controlled group as predicted in the ``OPTIMAL'' theory. Although the majority of experiments that included a measure related to perceived autonomy reported no group differences, the ``absence of evidence is not evidence of absence'' (Altman \& Bland, 1995). Thus, self-controlled practice conditions could be an autonomy-supportive manipulation but such an effect may actually be quite small and require much larger sample sizes to detect than those used in previous experiments (e.g., Barros et al., 2019; Ste-Marie et al., 2013) and in motor learning experiments in general (see Lohse, Buchanan, \& Miller, 2016 for a discussion). This argument of underpowered experimental designs is supported by the results of McKay and Ste-Marie (2020b) as these authors had one of the largest sample sizes to date in the self-controlled literature.
\end{quote}

\RC{\emph{Moreover, there was no explicit mention as to how levels of perceived autonomy might have a mechanistic role in motor skill learning. Self-control conditions might lead to higher levels of perceived autonomy but no learning advantage (as seen in the present study). But, when looking at the title of the paper and how the discussion unfolds, the authors shift their focus to claim that autonomy-support does not benefit motor learning.
}}

We agree that self-controlled practice conditions might lead to higher levels of perceived autonomy without any learning advantage. This, however, is not what is predicted in ``OPTIMAL'' theory where a causal relationship is put forward. The title of our paper and how the discussion unfolds is based on the results of our experiment, which suggests that increased perceptions of autonomy do not benefit motor learning. To your first point, we have added the following to the introduction to better outline how autonomy-support might have a mechanistic role in motor performance and learning according to the ``OPTIMAL'' theory of motor learning \textbf{(see Pages XX, Lines XX)}:

\begin{quote}
Within their ``OPTIMAL'' theory of motor learning, Wulf and Lewthwaite (2016) argued that providing learners with opportunities for choice creates an autonomy-supportive practice environment, which facilitates motor performance and learning. Specifically, the authors predict that autonomy-support facilitates performance by enhancing expectancies (Prediction 3, p.~1404), that enhanced expectancies and autonomy support contribute to efficient goal-action coupling by readying the motor system for task execution (Prediction 2, p.~1404), and that enhanced expectancies and autonomy support facilitate motor learning by making dopamine available for memory consolidation and neural pathway development (Prediction 7, p.~1404). In other words, these psychological benefits of increased perceptions of autonomy and competence, and the resulting increases in performance and learning are a by-product of having choice itself. Overall, Wulf and Lewthwaite (2016)'s ``OPTIMAL'' theory of motor learning provides a motivational explanation for the learning advantages of self-controlled practice conditions over yoked practice conditions.
\end{quote}

\RC{\emph{The narrative in its current form lacks clarity. Are the authors interested in detecting group differences in levels of perceived autonomy (regardless of whether it predicts learning) or in the role of autonomy-support in enhancing learning (i.e., replicate the self-control learning benefit under a motivational perspective)? The issue gets even more cloudy as the authors raise two methodological concerns identified in the autonomy-support literature: 1) "The first of these is that in self-controlled motor learning experiments participants in the self-controlled group are usually given choice over a single component of their practice and participants in the yoked group are not given choice over this component. However, within the context of practice itself there are many other opportunities for choice that participants may explore, independent of their assigned group" [Page6, line 4], and 2) "The other consideration relates to the instructions that are provided to participants in the yoked group [Page 6, line 16]", "Participants in yoked groups are not even aware that they have been denied an opportunity for choice, nor that their feedback schedule was created by another participant who was given choice over when feedback was or was not provided" [Page 6, line 20]. Regarding the first one, the benefits of self-control experiments have been found even when choice is given over an irrelevant aspect of the practice condition (incidental choice - Wulf \& Lewthwaite, 2016). Following this logic, it seems like the authors are referring specifically to the levels of perceived autonomy (having control over more variables might lead to higher levels of perceived autonomy). However, making yoked participants aware of being denied choice over some aspects of the practice condition does not prevent them from experiencing choice through other opportunities, weaking this argument. Moreover, this manipulation (i.e., making yoked participants aware of being denied choice) is better fitted under the controlling instructional language manipulation, which is contrasted against autonomy-supportive language (e.g., Hooyman et al., 2014). Yet, the authors do not even mention past research investigating the effect of controlling language on motor learning.
}}

Based on the ``OPTIMAL'' theory, we were interested in replicating the typical self-controlled learning advantage (group difference in retention favouring the self-controlled group) and testing the claim that self-controlled practice conditions are a way to create an autonomy-supportive environment (group difference in self-reported perceptions of autonomy). We have added details about our predictions to the revised manuscript (but we include this information below with your specific comment about these missing from the manuscript). In terms of the first methodological limitation, yes some researchers have reported a learning benefit of task-irrelevant choices (e.g., Lewthwaite \& Wulf 2015); however, this is a fairly selective stance given the mixed results surrounding task-irrelevant choices (e.g., Carter \& Ste-Marie 2017; Grand et al.~2015; McKay \& Ste-Marie 2020ab). In McKay \& Ste-Marie 2020a, the task-irrelevant choice groups actually performed with significantly more error in the acquisition and transfer.

\hypertarget{reviewer-3}{%
\section{Reviewer \#3}\label{reviewer-3}}

\RC{\emph{It was a pleasure to read and review your article "Increased perceptions of autonomy through choice fail to enhance motor skill retention" submitted to the Journal of Experimental Psychology: Human Perception and Performance. Your writing is apparent and concise. This article examines autonomy-support during the speed cup-stacking task. There were three conditions: the self-control group, the traditional yoked group, and the explicit yoked group. The results suggest that there are no statistically significant learning differences between the groups.
}}

This is our response.

\begin{quote}
This is a section quoted from the revised manuscript to illustrate the change.
\end{quote}

\clearpage

\hypertarget{references}{%
\section{References}\label{references}}

\begingroup

\hypertarget{refs}{}
\begin{CSLReferences}{0}{0}
\end{CSLReferences}

\endgroup


\end{document}\grid
